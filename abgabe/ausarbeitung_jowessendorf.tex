% Hinweise:
% - Dateinamen anpassen!
% - Name, Vorname, Studiengang, Matrikelnummer anpassen!
% - Hinweis zu Umlauten beachten (s.u.)!
% - Spellchecker verwenden!
% - PDF erzeugen: pdflatex ausarbeitung_emustermann.tex

\documentclass[12pt,a4paper]{article}
\usepackage[onehalfspacing]{setspace}
\topmargin=-2cm
\textwidth=16cm
\textheight=23cm
% LRTB
%\marginsize{3.0cm}{3.0cm}{2.0cm}{2.0cm}
\usepackage[german,english]{babel}
\usepackage{times}
\usepackage[utf8]{inputenc}
\usepackage{layouts}
\usepackage{url}
\begin{document}
\selectlanguage{german}
\newcommand{\3}{\ss}

\begin{center}
  \noindent{\large\bf%
    Ausarbeitung Betriebssysteme Sommersemester 2022}\\[1cm]
  \noindent{\Large\bf%
    Parallelisierung des KNN-Verfahrens}\\[3cm]
\end{center}

\noindent{\bf Name:} Mustermann\\

\noindent{\bf Vorname:} Erika\\

\noindent{\bf Studiengang:} Angewandte Scholastik\\

\noindent{\bf Matrikelnummer:} 0123456789\\[3cm]

\noindent Hiermit erkläre ich, dass ich das Programm und die
vorliegende Ausarbeitung selbstständig verfasst habe. Ich habe keine
anderen Quellen als die angegebenen benutzt und habe die Stellen in
Programm und in der Ausarbeitung, die anderen Quellen entnommen
wurden, in jedem Fall unter Angabe der Quelle als Entlehnung kenntlich
gemacht. Diese Erklärung gilt auch ohne meine Unterschrift, sobald ich
das Programm und die Ausarbeitung über die E-Prüfung der Vorlesung
Betriebssysteme im Lernraum des eKVV an der Universität Bielefeld
und unter Angabe meiner Matrikelnummer in der Ausarbeitung eingereicht
habe.

\thispagestyle{empty}
\newpage
\setcounter{page}{1}

Hinweis: ``So macht man Anführungszeichen'', "aber so geht es nicht".

\section{Thread-Pool}

Eins, zwo drei vier fünf, sechs sieben acht neun.
Eins, zwo drei vier fünf, sechs sieben acht neun.
Eins, zwo drei vier fünf, sechs sieben acht neun.
Eins, zwo drei vier fünf, sechs sieben acht neun.
Eins, zwo drei vier fünf, sechs sieben acht neun.
Eins, zwo drei vier fünf, sechs sieben acht neun.
Eins, zwo drei vier fünf, sechs sieben acht neun.
Eins, zwo drei vier fünf, sechs sieben acht neun.
Eins, zwo drei vier fünf, sechs sieben acht neun.
Eins, zwo drei vier fünf, sechs sieben acht neun.
Eins, zwo drei vier fünf, sechs sieben acht neun.
Eins, zwo drei vier fünf, sechs sieben acht neun.
Eins, zwo drei vier fünf, sechs sieben acht neun.
Eins, zwo drei vier fünf, sechs sieben acht neun.
Eins, zwo drei vier fünf, sechs sieben acht neun.
Eins, zwo drei vier fünf, sechs sieben acht neun.
Eins, zwo drei vier fünf, sechs sieben acht neun.
Eins, zwo drei vier fünf, sechs sieben acht neun.
Eins, zwo drei vier fünf, sechs sieben acht neun.
Eins, zwo drei vier fünf, sechs sieben acht neun.

\section{Ablauf}

Eins, zwo drei vier fünf, sechs sieben acht neun.
Eins, zwo drei vier fünf, sechs sieben acht neun.
Eins, zwo drei vier fünf, sechs sieben acht neun.
Eins, zwo drei vier fünf, sechs sieben acht neun.
Eins, zwo drei vier fünf, sechs sieben acht neun.
Eins, zwo drei vier fünf, sechs sieben acht neun.
Eins, zwo drei vier fünf, sechs sieben acht neun.
Eins, zwo drei vier fünf, sechs sieben acht neun.
Eins, zwo drei vier fünf, sechs sieben acht neun.
Eins, zwo drei vier fünf, sechs sieben acht neun.
Eins, zwo drei vier fünf, sechs sieben acht neun.
Eins, zwo drei vier fünf, sechs sieben acht neun.
Eins, zwo drei vier fünf, sechs sieben acht neun.
Eins, zwo drei vier fünf, sechs sieben acht neun.
Eins, zwo drei vier fünf, sechs sieben acht neun.
Eins, zwo drei vier fünf, sechs sieben acht neun.
Eins, zwo drei vier fünf, sechs sieben acht neun.
Eins, zwo drei vier fünf, sechs sieben acht neun.
Eins, zwo drei vier fünf, sechs sieben acht neun.
Eins, zwo drei vier fünf, sechs sieben acht neun.

Eins, zwo drei vier fünf, sechs sieben acht neun.
Eins, zwo drei vier fünf, sechs sieben acht neun.
Eins, zwo drei vier fünf, sechs sieben acht neun.
Eins, zwo drei vier fünf, sechs sieben acht neun.
Eins, zwo drei vier fünf, sechs sieben acht neun.
Eins, zwo drei vier fünf, sechs sieben acht neun.
Eins, zwo drei vier fünf, sechs sieben acht neun.
Eins, zwo drei vier fünf, sechs sieben acht neun.
Eins, zwo drei vier fünf, sechs sieben acht neun.
Eins, zwo drei vier fünf, sechs sieben acht neun.
Eins, zwo drei vier fünf, sechs sieben acht neun.
Eins, zwo drei vier fünf, sechs sieben acht neun.
Eins, zwo drei vier fünf, sechs sieben acht neun.
Eins, zwo drei vier fünf, sechs sieben acht neun.
Eins, zwo drei vier fünf, sechs sieben acht neun.
Eins, zwo drei vier fünf, sechs sieben acht neun.
Eins, zwo drei vier fünf, sechs sieben acht neun.
Eins, zwo drei vier fünf, sechs sieben acht neun.
Eins, zwo drei vier fünf, sechs sieben acht neun.
Eins, zwo drei vier fünf, sechs sieben acht neun.

\section{Aufteilung innerhalb der Berechnungsphasen}

Eins, zwo drei vier fünf, sechs sieben acht neun.
Eins, zwo drei vier fünf, sechs sieben acht neun.
Eins, zwo drei vier fünf, sechs sieben acht neun.
Eins, zwo drei vier fünf, sechs sieben acht neun.
Eins, zwo drei vier fünf, sechs sieben acht neun.
Eins, zwo drei vier fünf, sechs sieben acht neun.
Eins, zwo drei vier fünf, sechs sieben acht neun.
Eins, zwo drei vier fünf, sechs sieben acht neun.
Eins, zwo drei vier fünf, sechs sieben acht neun.
Eins, zwo drei vier fünf, sechs sieben acht neun.
Eins, zwo drei vier fünf, sechs sieben acht neun.
Eins, zwo drei vier fünf, sechs sieben acht neun.
Eins, zwo drei vier fünf, sechs sieben acht neun.
Eins, zwo drei vier fünf, sechs sieben acht neun.
Eins, zwo drei vier fünf, sechs sieben acht neun.
Eins, zwo drei vier fünf, sechs sieben acht neun.
Eins, zwo drei vier fünf, sechs sieben acht neun.
Eins, zwo drei vier fünf, sechs sieben acht neun.
Eins, zwo drei vier fünf, sechs sieben acht neun.
Eins, zwo drei vier fünf, sechs sieben acht neun.

\section{Koordination der Berechnungsphasen}

Eins, zwo drei vier fünf, sechs sieben acht neun.
Eins, zwo drei vier fünf, sechs sieben acht neun.
Eins, zwo drei vier fünf, sechs sieben acht neun.
Eins, zwo drei vier fünf, sechs sieben acht neun.
Eins, zwo drei vier fünf, sechs sieben acht neun.
Eins, zwo drei vier fünf, sechs sieben acht neun.
Eins, zwo drei vier fünf, sechs sieben acht neun.
Eins, zwo drei vier fünf, sechs sieben acht neun.
Eins, zwo drei vier fünf, sechs sieben acht neun.
Eins, zwo drei vier fünf, sechs sieben acht neun.
Eins, zwo drei vier fünf, sechs sieben acht neun.
Eins, zwo drei vier fünf, sechs sieben acht neun.
Eins, zwo drei vier fünf, sechs sieben acht neun.
Eins, zwo drei vier fünf, sechs sieben acht neun.
Eins, zwo drei vier fünf, sechs sieben acht neun.
Eins, zwo drei vier fünf, sechs sieben acht neun.
Eins, zwo drei vier fünf, sechs sieben acht neun.
Eins, zwo drei vier fünf, sechs sieben acht neun.
Eins, zwo drei vier fünf, sechs sieben acht neun.
Eins, zwo drei vier fünf, sechs sieben acht neun.

Eins, zwo drei vier fünf, sechs sieben acht neun.
Eins, zwo drei vier fünf, sechs sieben acht neun.
Eins, zwo drei vier fünf, sechs sieben acht neun.
Eins, zwo drei vier fünf, sechs sieben acht neun.
Eins, zwo drei vier fünf, sechs sieben acht neun.
Eins, zwo drei vier fünf, sechs sieben acht neun.
Eins, zwo drei vier fünf, sechs sieben acht neun.
Eins, zwo drei vier fünf, sechs sieben acht neun.
Eins, zwo drei vier fünf, sechs sieben acht neun.
Eins, zwo drei vier fünf, sechs sieben acht neun.
Eins, zwo drei vier fünf, sechs sieben acht neun.
Eins, zwo drei vier fünf, sechs sieben acht neun.
Eins, zwo drei vier fünf, sechs sieben acht neun.
Eins, zwo drei vier fünf, sechs sieben acht neun.
Eins, zwo drei vier fünf, sechs sieben acht neun.
Eins, zwo drei vier fünf, sechs sieben acht neun.
Eins, zwo drei vier fünf, sechs sieben acht neun.
Eins, zwo drei vier fünf, sechs sieben acht neun.
Eins, zwo drei vier fünf, sechs sieben acht neun.
Eins, zwo drei vier fünf, sechs sieben acht neun.

Eins, zwo drei vier fünf, sechs sieben acht neun.
Eins, zwo drei vier fünf, sechs sieben acht neun.
Eins, zwo drei vier fünf, sechs sieben acht neun.
Eins, zwo drei vier fünf, sechs sieben acht neun.
Eins, zwo drei vier fünf, sechs sieben acht neun.
Eins, zwo drei vier fünf, sechs sieben acht neun.
Eins, zwo drei vier fünf, sechs sieben acht neun.
Eins, zwo drei vier fünf, sechs sieben acht neun.
Eins, zwo drei vier fünf, sechs sieben acht neun.
Eins, zwo drei vier fünf, sechs sieben acht neun.
Eins, zwo drei vier fünf, sechs sieben acht neun.
Eins, zwo drei vier fünf, sechs sieben acht neun.
Eins, zwo drei vier fünf, sechs sieben acht neun.
Eins, zwo drei vier fünf, sechs sieben acht neun.
Eins, zwo drei vier fünf, sechs sieben acht neun.
Eins, zwo drei vier fünf, sechs sieben acht neun.
Eins, zwo drei vier fünf, sechs sieben acht neun.
Eins, zwo drei vier fünf, sechs sieben acht neun.
Eins, zwo drei vier fünf, sechs sieben acht neun.
Eins, zwo drei vier fünf, sechs sieben acht neun.

\section{Schwachstellen und Verbesserungsmöglichkeiten}

Eins, zwo drei vier fünf, sechs sieben acht neun.
Eins, zwo drei vier fünf, sechs sieben acht neun.
Eins, zwo drei vier fünf, sechs sieben acht neun.
Eins, zwo drei vier fünf, sechs sieben acht neun.
Eins, zwo drei vier fünf, sechs sieben acht neun.
Eins, zwo drei vier fünf, sechs sieben acht neun.
Eins, zwo drei vier fünf, sechs sieben acht neun.
Eins, zwo drei vier fünf, sechs sieben acht neun.
Eins, zwo drei vier fünf, sechs sieben acht neun.
Eins, zwo drei vier fünf, sechs sieben acht neun.
Eins, zwo drei vier fünf, sechs sieben acht neun.
Eins, zwo drei vier fünf, sechs sieben acht neun.
Eins, zwo drei vier fünf, sechs sieben acht neun.
Eins, zwo drei vier fünf, sechs sieben acht neun.
Eins, zwo drei vier fünf, sechs sieben acht neun.
Eins, zwo drei vier fünf, sechs sieben acht neun.
Eins, zwo drei vier fünf, sechs sieben acht neun.
Eins, zwo drei vier fünf, sechs sieben acht neun.
Eins, zwo drei vier fünf, sechs sieben acht neun.
Eins, zwo drei vier fünf, sechs sieben acht neun.

\section{Analyse der Laufzeiten}

Eins, zwo drei vier fünf, sechs sieben acht neun.
Eins, zwo drei vier fünf, sechs sieben acht neun.
Eins, zwo drei vier fünf, sechs sieben acht neun.
Eins, zwo drei vier fünf, sechs sieben acht neun.
Eins, zwo drei vier fünf, sechs sieben acht neun.
Eins, zwo drei vier fünf, sechs sieben acht neun.
Eins, zwo drei vier fünf, sechs sieben acht neun.
Eins, zwo drei vier fünf, sechs sieben acht neun.
Eins, zwo drei vier fünf, sechs sieben acht neun.
Eins, zwo drei vier fünf, sechs sieben acht neun.
Eins, zwo drei vier fünf, sechs sieben acht neun.
Eins, zwo drei vier fünf, sechs sieben acht neun.
Eins, zwo drei vier fünf, sechs sieben acht neun.
Eins, zwo drei vier fünf, sechs sieben acht neun.
Eins, zwo drei vier fünf, sechs sieben acht neun.
Eins, zwo drei vier fünf, sechs sieben acht neun.
Eins, zwo drei vier fünf, sechs sieben acht neun.
Eins, zwo drei vier fünf, sechs sieben acht neun.
Eins, zwo drei vier fünf, sechs sieben acht neun.
Eins, zwo drei vier fünf, sechs sieben acht neun.

\section{Begrenzende Faktoren}

Eins, zwo drei vier fünf, sechs sieben acht neun.
Eins, zwo drei vier fünf, sechs sieben acht neun.
Eins, zwo drei vier fünf, sechs sieben acht neun.
Eins, zwo drei vier fünf, sechs sieben acht neun.
Eins, zwo drei vier fünf, sechs sieben acht neun.
Eins, zwo drei vier fünf, sechs sieben acht neun.
Eins, zwo drei vier fünf, sechs sieben acht neun.
Eins, zwo drei vier fünf, sechs sieben acht neun.
Eins, zwo drei vier fünf, sechs sieben acht neun.
Eins, zwo drei vier fünf, sechs sieben acht neun.
Eins, zwo drei vier fünf, sechs sieben acht neun.
Eins, zwo drei vier fünf, sechs sieben acht neun.
Eins, zwo drei vier fünf, sechs sieben acht neun.
Eins, zwo drei vier fünf, sechs sieben acht neun.
Eins, zwo drei vier fünf, sechs sieben acht neun.
Eins, zwo drei vier fünf, sechs sieben acht neun.
Eins, zwo drei vier fünf, sechs sieben acht neun.
Eins, zwo drei vier fünf, sechs sieben acht neun.
Eins, zwo drei vier fünf, sechs sieben acht neun.
Eins, zwo drei vier fünf, sechs sieben acht neun.

\noindent Dies kann gelöscht werden:

\noindent textwidth in cm: \printinunitsof{cm}\prntlen{\textwidth}

\noindent textheight in cm: \printinunitsof{cm}\prntlen{\textheight}

\newpage
\thispagestyle{empty}

\section*{Quellen}

{\parindent0pt%
  
[1] Möller, Ralf: Skript Betriebssysteme, AG Technische Informatik,
Technische Fakultät, Universität Bielefeld, Sommersemester 2022

[2] Einstein, Albert: Zur Elektrodynamik bewegter K{\"o}rper. Annalen
der Physik 322(10):891-921, 1905

[3] Wikipedia entry ``K-nearest neighbors algorithm'', konsultiert am
20. Juli 2022,
\url{https://en.wikipedia.org/wiki/K-nearest_neighbors_algorithm}

}
\end{document}
